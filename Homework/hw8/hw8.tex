% ---------
%  Compile with "pdflatex hw1".
% --------
%!TEX TS-program = pdflatex
%!TEX encoding = UTF-8 Unicode

% Template borrowed from Jeff Erickson.

\documentclass[11pt]{article}
\usepackage[utf8]{inputenc}		% Allow some non-ASCII Unicode in source
\usepackage{jeffe, handout,graphicx}
\usepackage{forest}
\usepackage{mathtools}
\usepackage{subfigure}
\usepackage{pgffor}
\usepackage{etoolbox}
\setlength\parindent{0pt}
\newcommand{\bs}[1]{\boldsymbol{#1}}

\newcommand{\matmul}[2]{
	\bs{A}_{#1,0}\cdot\bs{B}_{0,#2}
	\foreach \i in {1, ..., 3}{
		+\bs{A}_{#1,\i}\cdot\bs{B}_{\i,#2}
	}
}

%letter, start
\newcommand{\StrassD}[3]{
	\numdef\lowR{#2}
	\numdef\highR{#2+1}
	\numdef\lowC{#3}
	\numdef\highC{#3+1}
	\begin{bmatrix}
		#1_{\lowR, \lowC} & #1_{\lowR, \highC} \\
		#1_{\highR, \lowC} & #1_{\highR, \highC} \\
	\end{bmatrix}
}

%matrix, startr, startc, letterA, letterB
\newcommand{\StrassM}[5]{
	\numdef\lowr{#2}
	\numdef\highr{#2+1}
	\numdef\lowc{#3}
	\numdef\highc{#3+1}
	\ifnumequal{#1}{1}{
		\left(#4_{\lowr,\lowc} + #4_{\highr,\highc}\right)
		\left(#5_{\lowr,\lowc}+#5_{\highr,\highc}\right)
	}{}
	\ifnumequal{#1}{2}{
		\left(#4_{\highr,\lowc} + #4_{\highr,\highc}\right)
		#5_{\lowr,\lowc}
	}{}
	\ifnumequal{#1}{3}{
		#4_{\lowr,\lowc}
		\left(#5_{\lowr,\highc}-#5_{\highr,\highc}\right)
	}{}
	\ifnumequal{#1}{4}{
		#4_{\highr,\highc}
		\left(#5_{\highr,\lowc}-#5_{\lowr,\lowc}\right)
	}{}
	\ifnumequal{#1}{5}{
		\left(#4_{\lowr,\lowc} + #4_{\lowr,\highc}\right)
		#5_{\highr,\highc}
	}{}
	\ifnumequal{#1}{6}{
		\left(#4_{\highr,\lowc} - #4_{\lowr,\lowc}\right)
		\left(#5_{\lowr,\lowc}+#5_{\lowr,\highc}\right)
	}{}
	\ifnumequal{#1}{7}{
		\left(#4_{\lowr,\highc} - #4_{\highr,\highc}\right)
		\left(#5_{\highr,\lowc}+#5_{\highr,\highc}\right)
	}{}
}

%matrix, start, startc, letterOut, letterIn
\newcommand{\StrassC}[5]{
	\numdef\lowr{#2}
	\numdef\highr{#2+1}
	\numdef\lowc{#3}
	\numdef\highc{#3+1}
	\ifnumequal{#1}{1}{
		#4_{\lowr,\lowc} &= #5_1 + #5_4 - #5_5 + #5_7
	}{}
	\ifnumequal{#1}{2}{
		#4_{\lowr,\highc} &= #5_3 + #5_5
	}{}
	\ifnumequal{#1}{3}{
		#4_{\highr,\lowc} &= #5_2 + #5_4
	}{}
	\ifnumequal{#1}{4}{
		#4_{\highr,\highc} &= #5_1 - #5_2 + #5_3 + #5_6
	}{}
}

%start, letterA, letterB
\newcommand{\StrassAll}[3]{
	\foreach \i in {1, ..., 7}{
		\bs{M}_\i &= \StrassM{\i}{#1}{#2}{#3}
	}
}


% =========================================================
%   Define common stuff for solution headers
% =========================================================
\Class{CS 6301.503}
\Semester{Spring 2019}
\Authors{1}
\AuthorOne{Scott C. Waggener}{scw180000}
%\Section{}

% =========================================================
\begin{document}
% ---------------------------------------------------------
\HomeworkHeader{8 (Implementation)}{1}
Readings
% ---------------------------------------------------------

\begin{solution}
	Complete
\end{solution}


% ---------------------------------------------------------
\HomeworkHeader{8 (Implementation)}{2}

Write out all of the terms of Strassen based matrix matrix multiplication
for $C = AB$ with BLAS dimensions $M = N = K = 4$ by applying the Strassen
decomposition twice (an initial decomposition then a recursive
decomposition)
% ---------------------------------------------------------


\begin{solution}
	Recall that the Strassen matrix decomposition algorithm defines the
	following new matrices.

	\begin{align*}
		&
		\begin{aligned}
			\bs{M}_1 &= \StrassM{1}{0}{0}{\bs{A}}{\bs{B}} \\
			\bs{M}_2 &= \StrassM{2}{0}{0}{\bs{A}}{\bs{B}} \\
			\bs{M}_3 &= \StrassM{3}{0}{0}{\bs{A}}{\bs{B}} \\
			\bs{M}_4 &= \StrassM{4}{0}{0}{\bs{A}}{\bs{B}} \\
			\bs{M}_5 &= \StrassM{5}{0}{0}{\bs{A}}{\bs{B}} \\
			\bs{M}_6 &= \StrassM{6}{0}{0}{\bs{A}}{\bs{B}} \\
			\bs{M}_7 &= \StrassM{7}{0}{0}{\bs{A}}{\bs{B}}
		\end{aligned}
		&\text{where}&
		&
		\begin{aligned}
			\StrassC{1}{0}{0}{\bs{C}}{\bs{M}} \\
			\StrassC{2}{0}{0}{\bs{C}}{\bs{M}} \\
			\StrassC{3}{0}{0}{\bs{C}}{\bs{M}} \\
			\StrassC{4}{0}{0}{\bs{C}}{\bs{M}}
		\end{aligned}
		&
	\end{align*}

	We begin by applying a partition over $\bs{C}$ into four block
	matrices

	\begin{align*}
		\bs{C} = \StrassD{\bs{C}}{0}{0}
	\end{align*}

	where the block matrices are given by

	\begin{align*}
		\begin{aligned}
			\bs{C}_{0, 0} &= \StrassD{A}{0}{0} \cdot \StrassD{B}{0}{0} \\
			\bs{C}_{0, 1} &= \StrassD{A}{0}{2} \cdot \StrassD{B}{0}{2}
		\end{aligned}
		&&
		\begin{aligned}
			\bs{C}_{1, 0} &= \StrassD{A}{2}{0} \cdot \StrassD{B}{2}{0} \\
			\bs{C}_{1, 1} &= \StrassD{A}{2}{2} \cdot \StrassD{B}{2}{2}
		\end{aligned}
	\end{align*}

	Next we recursively compute the matrix multiplications in this
	decomposition. Note the use of non-bold symbols reflecting that we
	are now working with scalar elements at this level of recursion.
	\newline

	\foreach \i in {0, 1}{
		\foreach \j in {0, 1}{
			\numdef\reali{2*\i}
			\numdef\realj{2*\j}

			For $\bs{C}_{\i, \j}$ we have

			\begin{align*}
				&
				\begin{aligned}
					M_1 &= \StrassM{1}{\reali}{\realj}{A}{B} \\
					M_2 &= \StrassM{2}{\reali}{\realj}{A}{B} \\
					M_3 &= \StrassM{3}{\reali}{\realj}{A}{B} \\
					M_4 &= \StrassM{4}{\reali}{\realj}{A}{B} \\
					M_5 &= \StrassM{5}{\reali}{\realj}{A}{B} \\
					M_6 &= \StrassM{6}{\reali}{\realj}{A}{B} \\
					M_7 &= \StrassM{7}{\reali}{\realj}{A}{B}
				\end{aligned}
				&\text{where}&
				&
				\begin{aligned}
					\StrassC{1}{\reali}{\realj}{C}{M} \\
					\StrassC{2}{\reali}{\realj}{C}{M} \\
					\StrassC{3}{\reali}{\realj}{C}{M} \\
					\StrassC{4}{\reali}{\realj}{C}{M}
				\end{aligned}
				&
			\end{align*}
		}
	}

\end{solution}


\end{document}
