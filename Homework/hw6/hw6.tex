% ---------
%  Compile with "pdflatex hw1".
% --------
%!TEX TS-program = pdflatex
%!TEX encoding = UTF-8 Unicode

% Template borrowed from Jeff Erickson.

\documentclass[11pt]{article}
\usepackage[utf8]{inputenc}		% Allow some non-ASCII Unicode in source
\usepackage{jeffe, handout,graphicx}
\usepackage{forest}
\usepackage{mathtools}
\usepackage{tikz, ifthen, etoolbox}
\usepackage{float}
\usetikzlibrary{positioning}
\newcommand{\pd}{\partial}
\newcommand{\bs}{\boldsymbol}
\newcommand{\ifstringequal}[4]{%
  \ifnum\pdfstrcmp{#1}{#2}=0
  #3%
  \else
  #4%
  \fi
}


% =========================================================
%   Define common stuff for solution headers
% =========================================================
\Class{CS 6301.503}
\Semester{Spring 2019}
\Authors{1}
\AuthorOne{Scott C. Waggener}{scw180000}
%\Section{}

% =========================================================
\begin{document}
\HomeworkHeader{6 (Design)}{1}% homework number, problem number
% ---------------------------------------------------------
Read
\begin{solution}
	Complete
\end{solution}

% =========================================================
\HomeworkHeader{6 (Design)}{2}% homework number, problem number
% ---------------------------------------------------------
Read
\begin{solution}
	Complete
\end{solution}

% =========================================================
\HomeworkHeader{6 (Design)}{3}% homework number, problem number
% ---------------------------------------------------------
Read
\begin{solution}
	Complete
\end{solution}

% =========================================================
\HomeworkHeader{6 (Design)}{4}% homework number, problem number
% ---------------------------------------------------------
Read
\begin{solution}
	Complete
\end{solution}

% =========================================================
\HomeworkHeader{6 (Design)}{5}% homework number, problem number
% ---------------------------------------------------------
Compute the receptive field size at the input to the global average pooling layer for ResNet 50.

\begin{solution}
	pass
\end{solution}
	Stride then add
\begin{proof}
	\begin{align}
		\bigg(
			\Big(
				\underbrace{1+2*3}_{\text{conv5}}
				+
				\underbrace{2*6}_{\text{conv4}}
				+
				\underbrace{2*4}_{\text{conv4}}
				+
				\underbrace{2*3}_{\text{conv4}}
			\Big)
			\underbrace{*2+2}_{\text{max pool}}
		\bigg)
		\underbrace{*2+6}_{\text{conv1}}
		\\
		=142
	\end{align}
\end{proof}

% =========================================================
\HomeworkHeader{6 (Design)}{6}% homework number, problem number
% ---------------------------------------------------------
\begin{enumerate}[(a)]
	\item How does the accuracy of the half wide version copare to the original
		version?
	\begin{solution}
		The accuracy at the final epoch for the full and half versions are
		\begin{align}
			\text{Full} &= 91.03\%
			&
			\text{Half} &= 89.31\%
		\end{align}
	\end{solution}
	\begin{proof}
	\end{proof}

	\item How long does an epoch of training take for both versions?
	\begin{solution}
		The times for epoch $3$ for both models (trained on GTX970) are
		\begin{align}
			\text{Full} &= 59s
			&
			\text{Half} &= 36s
		\end{align}
	\end{solution}
	\begin{proof}
	\end{proof}

	\item Approximate how feature map memory, filter memory, and compute change
		between the full and half width versions.
	\begin{solution}

		\begin{align}
			&\text{Feature map memory} = \frac{1}{2}
			\\
			&\text{Filter memory} = \frac{1}{2}
			\\
			&\text{Compute} = \frac{1}{4}
		\end{align}
	\end{solution}
	\begin{proof}
		As an approximation we can consider how CNN style 2D convolution layers
		will be impacted by the width reduction, as Resnet is primarily
		composed of such layers. For filter and feature map memory we will see
		a reduction approximately equal to the reduction in width. The
		dimensions of the filters and feature maps remain unchanged, but we are
		only considering half of the original feature maps and filters.
		\newline

		Recall that compute for CNN style 2D convolution depends on $N_o * N_i$
		which are both reduced by half, leading to a total reduction of $1/4$.
	\end{proof}

\end{enumerate}
% =========================================================
\HomeworkHeader{6 (Design)}{7}% homework number, problem number
% ---------------------------------------------------------
\begin{solution}
\end{solution}
\begin{proof}
\end{proof}
% =========================================================
\HomeworkHeader{6 (Design)}{8}% homework number, problem number
% ---------------------------------------------------------
\begin{solution}
\end{solution}
\begin{proof}
\end{proof}
% =========================================================
\HomeworkHeader{6 (Design)}{9}% homework number, problem number
% ---------------------------------------------------------
\begin{solution}
\end{solution}
\begin{proof}
\end{proof}
\end{document}
