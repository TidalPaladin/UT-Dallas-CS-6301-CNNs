% ---------
%  Compile with "pdflatex hw1".
% --------
%!TEX TS-program = pdflatex
%!TEX encoding = UTF-8 Unicode

% Template borrowed from Jeff Erickson.

\documentclass[11pt]{article}
\usepackage[utf8]{inputenc}		% Allow some non-ASCII Unicode in source
\usepackage{jeffe, handout,graphicx}
\usepackage{forest}

% =========================================================
%   Define common stuff for solution headers
% =========================================================
\Class{CS 6301.503}
\Semester{Spring 2019}
\Authors{1}
\AuthorOne{Scott C. Waggener}{scw180000}
%\Section{}

% =========================================================
\begin{document}
\HomeworkHeader{2 (Linear Algebra)}{1}% homework number, problem number
% ---------------------------------------------------------
Read: A guide to convolution arithmetic for deep learning.
\begin{solution}
	Complete
\end{solution}

% ---------------------------------------------------------
\HomeworkHeader{2 (Linear Algebra)}{2}% homework number, problem number
% ---------------------------------------------------------
\begin{enumerate}[(a)]\itemsep0pt
  \item
    What is the arithmetic intensity for matrix matrix multiplication with
		sizes $M, N, K$.

\begin{solution}
	\begin{align}
		\texttt{Compute} &= M N K \\
		\texttt{Spatial} &= KN + MK + MN
	\end{align}

	When multiplying MxK and KxN matrices we will produce a MxN output matrix
	where each element was the result of a linear combination of K elements.
	As such, we must compute M*N sums of K elements.
	Similarly, for memory operations we must bring in the original matrices and output
	the resulting matrix, leading to a sum over the dimensions of these matrices.
\end{solution}

  \item
	  Prove that arithmetic intensity for matrix matrix multiplication is
		maximized when $M=N=K$.
	\begin{proof}
		Choose $N$ such that it is the largest matrix dimension. Let $I(M, N,
		K)$ designate the total arithmetic intensity.
		We can express $M, K$ in terms of a factor of $N$.

		\begin{align}
			M = C_m N && K = C_k N && C_m, C_n \in \mathbb R^+
		\end{align}

		Using the result of part a we can express the total arithmetic
		intensity as a sum of the compute and memory components.

		\begin{align}
			I(M, N, K) &= M N K + \left(K N + M K + M N \right)\\
			& = C_m C_k N^3 + C_m C_k N^2 + C_m N^2 + C_k N^2
		\end{align}

		Since $N$ was chosen to be the largest of the terms, the coefficients
		$C_m, C_k$ must be on the interval

		\begin{equation}
			0 < C_m, C_k \leq 1
		\end{equation}

		From the form of $I(M, N, K)$ given above and the interval on which
		$C_m, C_k$ fall, it is clear that arithmetic intensity will be
		maximized for

		\begin{align}
			C_m = 1 && C_k = 1
		\end{align}

		So

		\begin{gather}
			\begin{align}
				M = C_m N = N && K = C_k N  = N
			\end{align}\\
			\therefore M=N=K
		\end{gather}
	\end{proof}
\end{enumerate}

% ---------------------------------------------------------
\HomeworkHeader{2 (Linear Algebra)}{3}% homework number, problem number
% ---------------------------------------------------------
What is the complexity (MACs and memory) of a $N_o = N_i$ dense layer applied
to vectorized versions of the following inputs:

\begin{enumerate}[(a)]\itemsep0pt
    \item MNIST $1 \times 28 \times 28$
\begin{solution}
	\begin{align}
		\texttt{Compute} &= \left(1*28*28\right)^2 = 6.15\E^{6} \\
		\texttt{Spatial} &= \left(1*28*28\right)^2 + 2\left(1*28*28\right) =
		6.16\E6
	\end{align}
	Explanation on next page
\end{solution}
    \item CIFAR $3 \times 32 \times 32$
\begin{solution}
	\begin{align}
		\texttt{Compute} &= \left(3*32*32\right)^2 = 9.44\E6 \\
		\texttt{Spatial} &= \left(3*32*32\right)^2 + 2\left(3*32*32\right) =
		9.44\E6
	\end{align}
\end{solution}
    \item ImageNet $3 \times 224 \times 224$
\begin{solution}
	\begin{align}
		\texttt{Compute} &= \left(3*224*224\right)^2 = 2.27\E10 \\
		\texttt{Spatial} &= \left(3*224*224\right)^2 + 2\left(3*224*224\right)
		= 2.27\E10
	\end{align}
\end{solution}
    \item Quasi 1/4 HD $3 \times 512 \times 1024$
\begin{solution}
	\begin{align}
		\texttt{Compute} &= \left(3*512*1024\right)^2 = 2.47\E12 \\
		\texttt{Spatial} &= \left(3*512*1024\right)^2 +
		2\left(3*512*1024\right) = 2.47\E12
	\end{align}
\end{solution}
    \item Quasi HD $3 \times 1024 \times 2048$
\begin{solution}
	\begin{align}
		\texttt{Compute} &= \left(3*1024*2048\right)^2 = 3.96\E13 \\
		\texttt{Spatial} &= \left(3*1024*2048\right)^2 +
		2\left(3*1024*2048\right) = 3.96\E13
	\end{align}
\end{solution}
\end{enumerate}

Vectorizing our input (using $L_d$ to denote feature map depth) gives us
\begin{align}
	N_i = L_d \times L_r \times L_c
\end{align}

The dense layer transforms an input to an output vector of the same length. As
such, our transformation matrix will be $N_i^2$. We are then left with matrix
vector multiplication with the following dimensions under BLAS notation.

\begin{align}
	M&=N_i & K&=N_i & N&=1 \\
	M&=L_d \times L_r \times L_c & K&=L_d \times L_r \times L_c & N&=1
\end{align}

From the result of Problem 2 we have

\begin{align}
	\texttt{Compute} &= M N K &
		\texttt{Spatial} &= KN + MK + MN \\
	&= \big(L_d \times L_r \times L_c\big)^2 &
		&= \big(L_d \times L_r \times L_c\big)^2 +  2* \big(L_d \times L_r \times L_c\big)
\end{align}

% ---------------------------------------------------------
\HomeworkHeader{2 (Linear Algebra)}{4}% homework number, problem number
% ---------------------------------------------------------
In practice, why can't you flatten a quasi HD input image to a vector for input
to a dense layer
% ---------------------------------------------------------

\begin{solution}

	The results of Problem 3 illustrate how arithmetic intensity for vectorized
	image inputs to a dense layer grows rapidly, reaching tens of trillions of
	operations. This level of arithmetic intensity for large inputs is
	impractical on today's hardware. Furthermore, the use of matrix
	vector multiplication (as opposed to matrix matrix for CNN style 2D
	convolution layers) creates a constant factor ratio of compute to memory
	intensity. This leads to a memory wall, as memory movement operations are
	much slower than compute operations. It is also worth noting that dense
	layers acting on a vectorized 3D input will have no information about the
	topology of the true input, and will be forced to look for relationships
	between features that are separated by a great distance in the original
	input.
\end{solution}

% ---------------------------------------------------------
\HomeworkHeader{2 (Linear Algebra)}{5}% homework number, problem number
% ---------------------------------------------------------
Can a dense layer trained on an input of $1024\times1$ be applied to an input
of size $2048\times1$ or $512\times1$
% ---------------------------------------------------------

\begin{solution}
	No to both. Dense layers learn weights such that a component in the output
	vector has a variable dependency on each of items in the input vector. The
	alterations needed to accommodate inputs of varying size would likely
	reduce the validity of the layer's output.
\end{solution}

% ---------------------------------------------------------
\HomeworkHeader{2 (Linear Algebra)}{6}% homework number, problem number
% ---------------------------------------------------------
Prove that CNN style 2D convolution with $N_o \times N_i \times F_r \times F_c$
filter, $N_i \times L_r \times L_c$ input and $N_o \times M_r \times M_c$
output can be lowered to the sum of $(F_r * F_c)$ matrix matrix multiplications.
% ---------------------------------------------------------

\begin{proof}
	Let the CNN style 2D convolution layer be of the usual form
	\begin{align}
		\boldsymbol{Y}^{3D} &= \boldsymbol{H}^{4D}\boldsymbol{X}^{3D} +
		\boldsymbol{V}^{3D}
	\end{align}

	If we consider a single output point $m_r \in M_r, m_c \in M_c$
	of a single output feature map $n_o \in N_o$
	we can express this as

	\begin{align}
		Y(n_o, m_r, m_c) &= \sum_{n_i}^{N_i}\sum_{f_r}^{F_r}\sum_{f_c}^{F_c}
		H(n_o, n_i, f_r, f_c)\ X(n_i, m_r + f_r, m_c + f_c)
	\end{align}

	If we expand the summation over $n_i$ only we get a vector vector
	multiplication as follows

	\begin{align}
		Y(n_o, m_r, m_c) &= \sum_{f_r}^{F_r}\sum_{f_c}^{F_c}
		\begin{bmatrix}
			H(n_o, 0, f_r, f_c) \\ H(n_o, 1, f_r, f_c) \\ \cdots \\ H(n_o, N_i, f_r, f_c)
		\end{bmatrix}^T
		\begin{bmatrix}
			X(0, f_r, f_c) \\ X(1, f_r, f_c) \\ \cdots \\ X(N_i, f_r, f_c)
		\end{bmatrix}
	\end{align}

	We must next generalize this to all output points of all feature maps. If
	we extend row vector $H$ into a matrix by adding $N_o$ rows for
	$n_o \in N_o$, and if we
	extend column vector $X$ into a matrix by adding $M_r \times M_c$ columns
	for the Cartesian product combinations of $m_r, m_c$, we get the following

	\begin{align}
		Y &= \sum_{f_r}^{F_r}\sum_{f_c}^{F_c}
		\underbrace{
			\begin{bmatrix}
				H(0, 0, f_r, f_c) &  \cdots &
					H(0, N_i,f_r, f_c) \\
				\vdots & \ddots & \vdots \\
				H(N_o, 0, f_r, f_c) &  \cdots &
				H(N_o, N_i,f_r, f_c)
			\end{bmatrix}
		}_{N_o \times N_i}
		\underbrace{
			\begin{bmatrix}
				X(0, f_r, f_c) & \cdots & X(0, f_r+M_r, f_c+M_c) \\
				\vdots & \ddots & \vdots \\
				X(N_i, f_r, f_c) & \cdots & X(N_i, f_r+M_r, f_c+M_c)
			\end{bmatrix}
		}_{N_i \times M_r * M_C}
	\end{align}

	From here we need only trivially reshape $Y$ from a 2D matrix into a $3D$
	tensor. The expression above indicates that CNN style 2D convolution can be
	lowered to a sum of $F_r * F_c$ matrix matrix multiplications. Furthermore,
	this result suggests that there are many alternative lowerings possible
	where each is determined by the which summations are expanded.

\end{proof}

% ---------------------------------------------------------
\HomeworkHeader{2 (Linear Algebra)}{7}% homework number, problem number
% ---------------------------------------------------------
How many MACs are required to compute each output point in a CNN style 2D
convolution layer $N_o \times N_i \times F_r \times F_c$
% ---------------------------------------------------------

\begin{solution}
	In an intermediate step of Problem 6 we expressed a single output point as
	a summation as follows

	\begin{align}
		Y(n_o, m_r, m_c) &= \sum_{n_i}^{N_i}\sum_{f_r}^{F_r}\sum_{f_c}^{F_c}
		H(n_o, n_i, f_r, f_c)\ X(n_i, m_r + f_r, m_c + f_c)
	\end{align}

	Recall that this summation can be expanded to an inner product with vectors
	of length $N_i * F_r * F_c$. We know trivially that the number of MACs for
	this operation will be given by

	\begin{align}
		\texttt{Compute}_1 &= F_r * F_c * N_i
	\end{align}

\end{solution}

% ---------------------------------------------------------
\HomeworkHeader{2 (Linear Algebra)}{8}% homework number, problem number
% ---------------------------------------------------------
How does CNN style 2D convolution complexity (MACs and memory) scale as a
function of
% ---------------------------------------------------------
\begin{enumerate}[(a)]\itemsep0pt
	\item Product of image rows and columns $(L_r*L_c)$
		\begin{solution}
			\begin{align}
				\texttt{Compute} \rightarrow \texttt{Linearly} \\
				\texttt{Memory} \rightarrow \texttt{Linearly} \\
			\end{align}
		\end{solution}
	\item Product of filter rows and columns $(F_r*F_c)$
		\begin{solution}
			\begin{align}
				\texttt{Compute} \rightarrow \texttt{Linearly} \\
				\texttt{Memory} \rightarrow \texttt{Linearly} \\
			\end{align}
		\end{solution}
	\item Product of input and output feature maps $(N_o * N_i)$
		\begin{solution}
			\begin{align}
				\texttt{Compute} \rightarrow \texttt{Linearly} \\
				\texttt{Memory} \rightarrow \texttt{Linearly} \\
			\end{align}
		\end{solution}
\end{enumerate}

From previous problems we know that MACs are determined by

\begin{align}
	\texttt{Compute} = N_i * F_r * F_c * N_o * M_r * M_c
\end{align}

and memory is determined by

\begin{align}
	\texttt{Memory} = N_i L_r L_c + N_o M_r M_c + N_i N_o F_r F_c
\end{align}

Both of these

% ---------------------------------------------------------
\HomeworkHeader{2 (Linear Algebra)}{9}% homework number, problem number
% ---------------------------------------------------------
Consider a CNN style 2D convolution layer with filter size $N_o \times N_i
\times F_r \times F_c$.
% ---------------------------------------------------------
\begin{enumerate}[(a)]\itemsep0pt
	\item How many padding $0$s such that output feature map is the same size
		as input?
		\begin{solution}

			Padding to maintain size across the CNN style 2D convolution layer
			is given by
			\begin{align}
				P_l+P_r&=F_c-1 & P_t+P_b&=F_r-1
			\end{align}

			This can easily be confirmed by realizing that
			\begin{align}
				M_r &= (L_r+P_r) - F_r + 1 \\
			\end{align}

			So total padding along rows is given by
			\begin{align}
				M_r = L_r \implies L_r &= L_r + P_r - F_r + 1 \\
				\implies P_r &= F_r - 1
			\end{align}
		\end{solution}
	\item What is the size of the border of $0$s for $F_r=F_c=1$?
		\begin{solution}
			$0$

			There is no need for padding since a $1\times1$ filter will map
			each point in the input feature maps to a point in an output
			feature map.
		\end{solution}
	\item What is the size of the border of $0$s for $F_r=F_c=3$?
		\begin{solution}
			\begin{align}
				P_l+P_r&=3-1 & P_t+P_b&=3-1 \\
				P_l=P_r&=1 & P_t=P_b&=1 \\
			\end{align}
		\end{solution}
	\item What is the size of the border of $0$s for $F_r=F_c=5$?
		\begin{solution}
			\begin{align}
				P_l+P_r&=5-1 & P_t+P_b&=5-1 \\
				P_l=P_r&=2 & P_t=P_b&=2 \\
			\end{align}
		\end{solution}
\end{enumerate}

% ---------------------------------------------------------
\HomeworkHeader{2 (Linear Algebra)}{10}% homework number, problem number
% ---------------------------------------------------------
Consider a CNN style 2D convolution layer with filter size $N_o \times N_i
\times F_r \times F_c$, input $N_i \times L_r \times L_c$, $P_r = F_r-1$ and
$P_c=F_c-1$.
% ---------------------------------------------------------

\begin{enumerate}[(a)]\itemsep0pt
	\item What is the size of the output feature map with striding $S_r=S_c=1$
		\begin{solution}
			\begin{align}
				M_r &= L_r
				&
				M_c &= L_c
			\end{align}

			Using a modified form of Relationship 6 in the reading we can write
			\begin{align}
				M_r &= \bigg\lfloor \frac{L_r + P_r - F_r}{S_r} \bigg\rfloor + 1
				\\
				&= \bigg\lfloor \frac{L_r + (F_r-1) - F_r}{S_r} \bigg\rfloor + 1
				\\
				&= \bigg\lfloor \frac{L_r - 1}{S_r} \bigg\rfloor + 1
			\end{align}

			Symmetry in the given input values allows this result to be
			extended to $M_c$. For a stride of one and padding given by filter
			dimension minus one, we get the expected result of $M_r = L_r$

		\end{solution}
	\item What is the size of the output feature map with striding $S_r=S_c=2$
		\begin{solution}
			\begin{align}
				M_r &= \bigg\lfloor \frac{L_r - 1}{2} \bigg\rfloor + 1
				&
				M_c &= \bigg\lfloor \frac{L_c - 1}{2} \bigg\rfloor + 1
			\end{align}

			We use the result of the previous part with $S_r = S_c = 2$


		\end{solution}
	\item How does this change the shape of the equivalent lowered matrix
		equation
		\begin{solution}
			In the equivalent lowered matrix matrix operation, striding alters
			the input matrix of shape $N_i \times (M_r * M_c)$.
			Values $M_r, M_c$ are divided by $S_r, S_c$ respectively. This in
			turn alters the shape of the output matrix in a similar way.
			Intuitively, striding is a downsampling operation where the input
			feature map is roughly downsampled by a factor of $S_r, S_c$
			along $L_r, L_c$ respectively.
		\end{solution}
\end{enumerate}

% ---------------------------------------------------------
\HomeworkHeader{2 (Linear Algebra)}{11}% homework number, problem number
% ---------------------------------------------------------
Can a CNN style 2D convolution layer trained on an input of size
$3 \times 1024 \times 2048$ be applied to an input of size:
% ---------------------------------------------------------

\begin{enumerate}[(a)]\itemsep0pt
	\item $3 \times 512 \times1024$
		\begin{solution}
			Yes, there are upsampling techniques like deconvolution or
			interpolation that could be used.

			This is the strength of CNN style 2D convolution layers. Since each
			output point is determined by input points exposed to the filter
			kernel at a given position, there will be no dependencies between
			distant points in the input feature map. Upsampling and
			downsampling can be applied to an input without compromising the
			validity of outputs from the CNN layer.
		\end{solution}
	\item $3\times 512 \times512$
		\begin{solution}
			Again, upsampling techniques could be used, however care must be
			taken since $U_r \neq U_c$.
		\end{solution}
\end{enumerate}

% ---------------------------------------------------------
\HomeworkHeader{2 (Linear Algebra)}{12}% homework number, problem number
% ---------------------------------------------------------
In a standard RNN, if the state update matrix is constrained to a diagonal,
what does this do for the mixing of the previous state with new inputs.
% ---------------------------------------------------------

\begin{solution}
	It forces an output element $y_i$ to depend only on the value of $y_i$ at
	time $t-1$, rather than all outputs at time $t-1$.
	\newline

	An RNN layer is given by

	\begin{align}
		\boldsymbol{y}_t = \boldsymbol{H}\boldsymbol{X} +
		\boldsymbol{G}\boldsymbol{y}_{t-1} + \boldsymbol{V}
	\end{align}

	The state update matrix $\boldsymbol{G}$ acts on vector
	$\boldsymbol{y}_{t-1}$ which was produced by the previous RNN state. For
	non-diagonal $\boldsymbol{G}$ this will produce a vector of length $N$
	where each element is a linear combination of all outputs from the previous
	state, as follows

	\begin{align}
		\boldsymbol{G}\boldsymbol{y}_{t-1} =
		\begin{pmatrix}
			G_{00}y^{t-1}_{0} + G_{01}y^{t-1}_{1} + \dots +
			G_{0 N_o}y^{t-1}_{N_o}
			\\
			G_{10}y^{t-1}_{0} + G_{11}y^{t-1}_{1} + \dots +
			G_{0 N_o}y^{t-1}_{N_o}
			\\
			\vdots
			\\
			G_{N_o 0}y^{t-1}_{0} + G_{N_o 1}y^{t-1}_{1} + \dots +
			G_{N_o N_o}y^{t-1}_{N_o}
		\end{pmatrix}
	\end{align}


	Constraining $\boldsymbol{G}$ to the diagonal means that a given output
	$y_i \in \boldsymbol{y}$ will be determined only be a scaled factor of $y_i
	\in \boldsymbol{y}_{t-1}$ as follows

	\begin{align}
		\boldsymbol{G}\boldsymbol{y}_{t-1} =
		\begin{pmatrix}
			G_{00}y^{t-1}_{0}
			\\
			G_{11}y^{t-1}_{1}
			\\
			\vdots
			\\
			G_{N_o N_o}y^{t-1}_{N_o}
		\end{pmatrix}
	\end{align}

	It is clear that in the addition of column vectors there will be no
	interplay between different output features.
\end{solution}

% ---------------------------------------------------------
\HomeworkHeader{2 (Linear Algebra)}{13}% homework number, problem number
% ---------------------------------------------------------
The size of the input to the global average pooling layer is
$1024\times16\times32$.
% ---------------------------------------------------------
\begin{enumerate}[(a)]\itemsep0pt
	\item What is the size of the output
		\begin{solution}
			$1024 \times 1$

			The global average pooling layer computes a global average for each
			of the input feature maps. As such, we will average
			feature maps of size $16\times32$ to a single value, repeated for
			each of the $1024$ given feature maps.
		\end{solution}
	\item What is the complexity (MACs) of the layer
		\begin{solution}
			\begin{align}
				\texttt{Compute} &= N_i * L_r * L_c \\
					&= 1024*16*32 \\
					&= 524288 \\
			\end{align}

			Global average pooling can be thought of as a matrix vector
			multiplication with the following BLAS dimensions

			\begin{align}
				M &= N_i & K &= L_r * L_c & N &= 1
			\end{align}

			where the $K\times N$ matrix has each value set to $1/(L_r * L_c)$
			for averaging. Based on this interpretation we can determine the
			complexity of the layer.

		\end{solution}
\end{enumerate}
\end{document}
